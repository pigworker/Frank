\documentclass{article}
\usepackage{times}
\usepackage{a4wide}
\usepackage{upgreek}
\usepackage{latexsym}
\usepackage{amsmath}
\usepackage{stmaryrd}
\usepackage{mathabx}

\parskip 0.1in
\parindent 0in

%%%%%%%%%%%%%%%%%%%%%%%%%%%%%%%%%%%%%%%%%%%%%%%%%%%%%%%%%%%%%%%%%%%%%%%%%%%%%
%%% Inference Rules (some ancient macros by Conor)                        %%%
%%%%%%%%%%%%%%%%%%%%%%%%%%%%%%%%%%%%%%%%%%%%%%%%%%%%%%%%%%%%%%%%%%%%%%%%%%%%%

\newlength{\rulevgap}
\setlength{\rulevgap}{0.05in}
\newlength{\ruleheight}
\newlength{\ruledepth}
\newsavebox{\rulebox}
\newlength{\GapLength}
\newcommand{\gap}[1]{\settowidth{\GapLength}{#1} \hspace*{\GapLength}}
\newcommand{\dotstep}[2]{\begin{tabular}[b]{@{}c@{}}
                         #1\\$\vdots$\\#2
                         \end{tabular}}
\newlength{\fracwid}
\newcommand{\dotfrac}[2]{\settowidth{\fracwid}{$\frac{#1}{#2}$}
                         \addtolength{\fracwid}{0.1in}
                         \begin{tabular}[b]{@{}c@{}}
                         $#1$\\
                         \parbox[c][0.02in][t]{\fracwid}{\dotfill} \\
                         $#2$\\
                         \end{tabular}}
\newcommand{\Rule}[2]{\savebox{\rulebox}[\width][b]                         %
                              {\( \frac{\raisebox{0in} {\( #1 \)}}       %
                                       {\raisebox{-0.03in}{\( #2 \)}} \)}   %
                      \settoheight{\ruleheight}{\usebox{\rulebox}}          %
                      \addtolength{\ruleheight}{\rulevgap}                  %
                      \settodepth{\ruledepth}{\usebox{\rulebox}}            %
                      \addtolength{\ruledepth}{\rulevgap}                   %
                      \raisebox{0in}[\ruleheight][\ruledepth]               %
                               {\usebox{\rulebox}}}
\newcommand{\Case}[2]{\savebox{\rulebox}[\width][b]                         %
                              {\( \dotfrac{\raisebox{0in} {\( #1 \)}}       %
                                       {\raisebox{-0.03in}{\( #2 \)}} \)}   %
                      \settoheight{\ruleheight}{\usebox{\rulebox}}          %
                      \addtolength{\ruleheight}{\rulevgap}                  %
                      \settodepth{\ruledepth}{\usebox{\rulebox}}            %
                      \addtolength{\ruledepth}{\rulevgap}                   %
                      \raisebox{0in}[\ruleheight][\ruledepth]               %
                               {\usebox{\rulebox}}}
\newcommand{\Axiom}[1]{\savebox{\rulebox}[\width][b]                        %
                               {$\frac{}{\raisebox{-0.03in}{$#1$}}$}        %
                      \settoheight{\ruleheight}{\usebox{\rulebox}}          %
                      \addtolength{\ruleheight}{\rulevgap}                  %
                      \settodepth{\ruledepth}{\usebox{\rulebox}}            %
                      \addtolength{\ruledepth}{\rulevgap}                   %
                      \raisebox{0in}[\ruleheight][\ruledepth]               %
                               {\usebox{\rulebox}}}
\newcommand{\RuleSide}[3]{\gap{\mbox{$#2$}} \hspace*{0.1in}               %
                            \Rule{#1}{#3}                          %
                          \hspace*{0.1in}\mbox{$#2$}}
\newcommand{\AxiomSide}[2]{\gap{\mbox{$#1$}} \hspace*{0.1in}              %
                             \Axiom{#2}                            %
                           \hspace*{0.1in}\mbox{$#1$}}
\newcommand{\RULE}[1]{\textbf{#1}}
\newcommand{\hg}{\hspace{0.2in}}

\begin{document}
\title{another crack at types for Frank}
\author{Conor}
\maketitle

\section{introduction}

Frank types are designed to separate the businesses of ensuring that values are compatible and that interactions are possible. Frank expressions have an operadic structure, with each operation having multiple \emph{ports} and one \emph{peg}. Ports and pegs each have an associated \emph{value} type: to put a peg in a port, their value types must be equal. Effect checking is relative to an \emph{ambient ability}, being a collection of distinct \emph{interfaces} offering services to an expression from its environment. A peg is labelled by its \emph{requirements} of the ambient ability; a port is labelled by its \emph{modification} of the ambient ability. Correspondingly, we generalize from $n$-adic functions with port-to-peg dataflow to handlers, with permissions flowing peg-to-ports. Crucially, effect annotations at ports characterize the requests which must be handled.

Let's have some types.
\newcommand{\ix}[2]{\ulcorner #2 \urcorner^{#1}}
\newcommand{\ixp}[2]{( #2 )^{#1}}
\[\begin{array}{lrrll}
\mbox{value type} &
V & ::= & \{C\} & \mbox{suspended computation} \\
  && | & D\:V^\ast & \mbox{datatype with actual parameters} \\
  && | & X & \mbox{type variable (flexible or rigid)}\\
\mbox{computation type} &
C & ::= & ([\rho]V)^\ast\:[\sigma]V  & \mbox{some ports and a peg}\\
\mbox{ability} &
\sigma & ::= & \emptyset & \mbox{purity} \\
   && | & \alpha & \mbox{abstract ambient} \\
   && | & \sigma, I\:V^\ast\quad\mbox{where $I\not\in\sigma$} & \mbox{distinct interface}\\
\mbox{adjustment} &
\rho & ::= & \emptyset & \mbox{purity} \\
   && | & \varepsilon & \mbox{identity} \\
   && | & \rho + I\:V^\ast & \mbox{extension}\\
   && | & \rho - I & \mbox{suppression}\\
\end{array}\]

It makes sense for an adjustment to suppress an interface, but it does not make sense
for an ability to have negative interfaces. Taking the trouble to separate adjustments from
abilities is the key to making the system make sense.


\section{abilities and adjustments}

Let us identify abilities up to permutation of their interfaces.
Let $\sigma-I$ be the ability given by removing any $I$ interface from $\sigma$, so that the operation
\[
\sigma + I\;\vec{V} = \sigma-I,I\:\vec{V}
\]
is well defined. Note that if $\sigma, I\:\vec{V}$ is a valid ability, then $\sigma + I\:\vec{V} = \sigma, I\:\vec{V}$.

Adjustments form a monoid, acting on abilities thus:
\[
\sigma + \varepsilon = \sigma \qquad
\sigma + \emptyset = \emptyset \qquad
\sigma + (\rho + I \:\vec{V}) = (\sigma + \rho) + I\:\vec{V} \qquad
\sigma + (\rho - I) = (\sigma + \rho) - I
\]
That is, the rightmost $I$-adjustment is the one which actually happens.

Abilities also enjoy a partial ordering generated as follows
\[
\Axiom{\emptyset\subseteq\Sigma} \qquad
\Axiom{\alpha\subseteq\alpha} \qquad
\Rule{\sigma\subseteq\sigma'}
  {\sigma, I\:\vec{V}\subseteq\sigma',I\:\vec{V}}
\]
Moreover, we may readily check that the adjustment action is monotonic. The
following is admissible:
\[
\Rule{\sigma_0\subseteq\sigma_1}{\sigma_0+\rho\subseteq\sigma_1+\rho}
\]
\textbf{Remark.}~The need for (right) monotonicity is why the ordering admits only $\alpha\subseteq\alpha$, with $\alpha\not\subseteq\alpha,I\:\vec{V}_0$. We need only consider
acting on both sides by $(\varepsilon,I\:\vec{V}_1)\cdot$, where $\vec{V}_0\neq\vec{V}_1$, to see dangerous
shadowing which would really bite us if we retain a `most local handler for a given operator' approach. A (desirable?) alternative approach is to work in the category whose objects are abilities and whose morphisms are ability embeddings, which might correspond operationally to the explicit passing of handler tables.

We may write the port annotation $[\varepsilon]$ as $[]$ or even omit it altogether. We may write the
peg annotation $[\alpha]$ as $[]$. It is acceptable to write the computation type $([]U) []V$ as $U\to V$.


\section{datatypes}

A datatype is given by a declaration
\[
\mathsf{data}\:D\:\vec{X}\; \{\ix{|i}{k_i \ixp{j}{U_{ij}}}\}
\]
That is, it is explicitly parametrized by variables standing for value types and given by a possibly empty collection of constructor templates. The same constructor name may be used in multiple datatypes.

For the moment, let us consider effect-monomorphic datatypes, requiring the abilities mentioned in any
computation types within the $U_{ij}$ to start with $\emptyset$, rather than $\alpha$. We can review this position once we start to add polymorphism.


\section{interfaces}

An interface is given by a declaration
\[
\mathsf{interface}\:I\:\vec{X}\; \{\ix{|i}{c_i \ixp{j}{U_{ij}}\:[] V_j}\}
\]
It is parametrized explicitly by value types and given by a collection of commands, each with any number of
parameter types and exactly one output type. Command names must be unique to a given interface.

As with datatypes, let us presume for the moment that $\alpha$ is nowhere to be found.


\section{monomorphic typing}

Frank's linguistic objects can be verbs or nouns. Accordingly, the contexts are given by
\newcommand{\EC}{\mathcal{E}}
\newcommand{\hab}{\!:\!}
\[\begin{array}{rrll}
\Gamma & ::= & \EC & \mbox{empty} \\
  & | & \Gamma, x\hab V & \mbox{adding a noun} \\
  & | & \Gamma, f\:C & \mbox{adding a verb}
\end{array}\]

Let's have some terms.
\[\begin{array}{rrrll}
\mbox{terms} &
t & ::= & u & \mbox{usage} \\
  && | & \{m^{|\ast}\} & \mbox{suspension} \\
  && | & k\:t^\ast & \mbox{construction} \\
\mbox{usages} &
u & ::= & x & \mbox{noun} \\
  && | & d\:t^\ast & \mbox{action} \\
\mbox{doers} &
d & ::= & f & \mbox{verb} \\
  && | & u! & \mbox{invocation} \\
  && | & c & \mbox{command} \\
\mbox{methods} &
m & ::= & r^\ast o & \mbox{matching} \\
\mbox{outcomes} &
o & ::= & \mapsto t & \mbox{returning} \\
  && | & \# x & \mbox{refutation} \\
\mbox{requests} &
r & ::= & p & \mbox{pattern} \\
  && | & c\:p^\ast ? f & \mbox{handler} \\
  && | & ? f & \mbox{yield} \\
\mbox{patterns} &
p & ::= & x & \mbox{naming} \\
  && | & k\:p^\ast & \mbox{inspection} \\
  && | & \{f\} & \mbox{preparation} \\
\end{array}\]

\newcommand{\turn}[1]{\left[#1\right]\!\!-}
\newcommand{\has}{\;\textsc{has}\;}
\newcommand{\is}{\;\textsc{is}\;}
\newcommand{\does}{\;\textsc{does}\;}
\newcommand{\allows}{\;\textsc{allows}\;}
\[\begin{array}{c}
\framebox{$\Gamma \turn\sigma u \is V$} \medskip\\

\Axiom{\Gamma,x\hab V,\Gamma' \turn\sigma x \is V} \\
\Rule{\Gamma \turn\sigma d\does \ixp{i}{[\rho_i]V_i}\:[\sigma']V\quad
  \sigma'\subseteq\sigma\quad
  \ix{i}{\Gamma \turn{\sigma+\rho_i} V_i \has t_i} }
  {\Gamma \turn\sigma d\:\ix{i}{t_i} \is V}
\end{array}\]

\[\begin{array}{c}
\framebox{$\Gamma \turn\sigma d \does C$} \medskip\\

\Axiom{\Gamma,f\:C,\Gamma' \turn\sigma f \does C} \qquad
\Rule{\Gamma \turn\sigma u \is \{C\}}
   {\Gamma \turn\sigma u! \does C}\\
\AxiomSide {c\:\ixp{i}{V_i}\:[]V' \in I\:\vec{U}}
 {\Gamma,\turn{\sigma} c \does \ixp{i}{V_i}\:[\emptyset,I\:\vec{U}]V'}\\
\end{array}\]

\[\begin{array}{c}
\framebox{$\Gamma \turn\sigma V \has t$} \medskip\\

\Rule{\Gamma\turn\sigma e \is V}
  {\Gamma\turn\sigma V \has e} \qquad
\Rule{\Gamma\vdash C \allows \vec{m}}
  {\Gamma\turn\sigma \{C\}\has \{\vec{m}\}}\\
\RuleSide{\ix{i}{\Gamma\turn\sigma V_i\has t_i}}{k\:\ixp{i}{U_i}\in D\:\vec{U}}
  {\Gamma\turn\sigma D\:\vec{U}\has k\:\ix{i}{t_i}}
\end{array}\]

The next judgement ensures that a collection of methods partitions the possible requests.
Request types have the grammar
\[\begin{array}{rrll}
R & ::= & ([\rho]V) & \mbox{port} \\
  & | & V/\rho & \mbox{value} \\
\end{array}\]
where the latter means that any permitted commands have been handled, but we retain
the information about what was permitted---this will prove useful for data polymorphism.

The first two rules allow us to permute method rows or request columns. The next three allow us to
split the first (hence, by permutation, any) request into cases: respectively, we may
handle some effects enabled in an adjustment, yield computation for effects not ready to be handled,
or do ordinary constructor case analysis. Then we need to extend
the context with nouns or verbs, and finally deliver an outcome.
\[\begin{array}{c}
\framebox{$\Gamma \vdash \vec{R}\:[\rho]V \allows \vec{m}$} \medskip\\

\Rule{\Gamma \vdash \ixp{i}{R_{\pi i}} [\sigma]V \allows \ix{j}{\ix{i}{r_{j(\pi i)}}\:o_j}}
  {\Gamma \vdash \ixp{i}{R_{i}} [\sigma]V \allows \ix{j}{\ix{i}{r_{ji}}\:o_j}} \qquad
\Rule{\Gamma \vdash \vec{R} [\sigma]V \allows \ix{j}{\vec{r}_{\pi j}\:o_{\pi j}}}
  {\Gamma \vdash \vec{R} [\sigma]V \allows \ix{j}{\vec{r}_j\:o_j}} \\
\Rule{\begin{array}{@{}ll@{}}
 &\Gamma \vdash
   U/\rho\:\vec{R}\:[\sigma]V \allows
 \ix{j}{p_{\bullet j}\:\vec{m}_{\bullet j}}
  \\
  \mbox{for each}\:c_i\:\ixp{j}{U_{ij}}\:[]V_i \in I\:\vec{W}\in\emptyset+\rho &
  \Gamma \vdash \ix{j}{U_{ij}}\:\{([]V_i)\:[\sigma]V\}\:\vec{R}\:[\sigma]V \allows
  \ix{j}{\vec{p}_{ij} \: \{f_{ij}\}\:\vec{m}_{ij}}
   \end{array}}
   {\Gamma \vdash ([\rho]U)\:\vec{R}\:[\sigma]V \allows
    \ix{j}{p_{\bullet j}\:\vec{m}_{\bullet j}}|
    \ix{i}{\ix{j}{(c_i\:\vec{p}_{ij} ? f_{ij})\:\vec{m}_{ij}}}}
\\
\Rule{\begin{array}{@{}l@{}}\Gamma \vdash 
 U/\rho\:\vec{R}\:[\sigma]V \allows
 \ix{j}{p_{\bullet j}\:\vec{m}_{\bullet j}}
  \\
  \Gamma \vdash \{[\sigma + \rho]U\}\:\vec{R}\:[\sigma]V \allows
  \ix{j}{\{f_{?j}\}\:\vec{m}_{?j}}
   \end{array}}
   {\Gamma \vdash ([\rho]U)\:\vec{R}\:[\sigma]V \allows
    \ix{j}{p_{\bullet j}\:\vec{m}_{\bullet j}}|
      \ix{j}{(? f_{?j})\:\vec{m}_{?j}}}
\\
\Rule{\mbox{for each}\:k_i\:\ixp{j}{U_{ij}} \in D\:\vec{V},
\Gamma \vdash \ix{j}{U_{ij}/\rho}\;\vec{R}\:[\sigma]V \allows \ix{j}{\vec{p}_{ij}\:\vec{m}_{ij}}
}
  {\Gamma \vdash (D\:\vec{V}/\rho)\:\vec{R}\:[\sigma]V
     \allows \ix{i}{\ix{j}{(k_i\:\vec{p}_{ij})\:\vec{m}_{ij}}}}
\\
\Rule{\Gamma, x \hab V \vdash \vec{R} \allows m}
  {\Gamma \vdash V/\rho\:\vec{R}\:[\sigma]V \allows x\:m}
\qquad
\Rule{\Gamma, f\:C \vdash \vec{R} \allows m}
  {\Gamma \vdash \{C\}/\rho\:\vec{R}\:[\sigma]V \allows \{f\}\:m}
\\
\Rule{\Gamma \turn\sigma V \has t}
  {\Gamma \vdash [\sigma]V \allows \mapsto t}
\qquad
\Axiom{\Gamma,x\hab D\:\vec{U},\Gamma' \vdash [\sigma]V \allows \# x }
 \;\;D\:\vec{U}\;\mbox{constructorless}
\end{array}\]

Note that the rule for handling commands ensures that only and exactly the commands
added by the port's adjustment to the ambient are executed.


\section{polymorphism}

An inspection of the rules shows that any valid judgment is preserved by substituting for its free
type variables, or by substituting any ability for $\alpha$. We may correspondingly consider constructions in the empty context to be polymorphic. Let us allow global definitions with type signatures $g\: C$ and
adopt the rule
\[
\Axiom{\Gamma\turn\sigma g\does C|\vec{X}:=\vec{V},\alpha:=\sigma}
\]

In turn, that would rather suggest that we can make datatypes silently effect-polymorphic and also
allow existential type variables in constructor types (being those which are not parameters). We should
now keep rigid type variables in contexts. The constructor rule becomes
\[
\RuleSide{\ix{i}{\Gamma\turn\sigma U_i|\vec{Y}:=\vec{W},\alpha:=\sigma\has t_i}}
  {\exists\vec{Y}.\:k\:\ixp{i}{U_i}\in D\:\vec{V}}
  {\Gamma\turn\sigma D\:\vec{V}\has k\:\ix{i}{t_i}}
\]
For case splitting, retaining the adjustment information for value requests tells us the ambient ability
for the construction of the data.
\[
\Rule{\mbox{for each}\:\exists \vec{Y}.\: k_i\:\ixp{j}{U_{ij}} \in D\:\vec{V},\;
  \Gamma,\vec{Y} \vdash \ix{j}{U_{ij}|\alpha:=\sigma+\rho}\;\vec{R}\:[\sigma]V \allows    \ix{j}{\vec{p}_{ij}\:\vec{m}_{ij}}}
 {\Gamma \vdash (D\:\vec{V}/\rho)\:\vec{R}\:[\sigma]V
   \allows \ix{i}{\ix{j}{(k_i\:\vec{p}_{ij})\:\vec{m}_{ij}}}}
\]

And we can play exactly the same game with interfaces and commands.
\[
\AxiomSide {\exists\vec{Y}.\:c\:\ixp{i}{V_i}\:[]V' \in I\:\vec{U}}
 {\Gamma,\turn{\sigma} c \does (\ixp{i}{V_i}\:[\emptyset,I\:\vec{U}]V')|\vec{Y}:=\vec{W},\alpha:=\sigma}\\
\]
\[
\Rule{\begin{array}{@{}ll@{}}
 &\Gamma \vdash
   U/\rho\:\vec{R}\:[\sigma]V \allows
 \ix{j}{p_{\bullet j}\:\vec{m}_{\bullet j}}
  \\
  \mbox{for each}\:\exists{\vec{Y}}.\:c_i\:\ixp{j}{U_{ij}}\:[]V_i \in I\:\vec{W}\in\emptyset+\rho &
  \Gamma,\vec{Y} \vdash \ix{j}{U_{ij}|\alpha:=\sigma+\rho}\:\{([]V_i |\alpha:=\sigma+\rho)\:[\sigma]V\}
  \:\vec{R}\:[\sigma]V\\
& \qquad\qquad \allows
  \ix{j}{\vec{p}_{ij} \: \{f_{ij}\}\:\vec{m}_{ij}}
   \end{array}}
   {\Gamma \vdash ([\rho]U)\:\vec{R}\:[\sigma]V \allows
    \ix{j}{p_{\bullet j}\:\vec{m}_{\bullet j}}|
    \ix{i}{\ix{j}{(c_i\:\vec{p}_{ij} ? f_{ij})\:\vec{m}_{ij}}}}
\]

That is, the polymorphic $\alpha$s in the command types get instantiated with the ambient ability for the command's invocation.

\end{document}